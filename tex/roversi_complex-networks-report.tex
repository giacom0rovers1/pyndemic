%%%%%%%%%%%%%%%%%%%%%%%%%%%%%%%%%%%%%%%%%
% Short Sectioned Assignment
% LaTeX Template
% Version 1.1 (26/01/19)
%
% This template has been downloaded from:
% http://www.LaTeXTemplates.com
%
% Original author:
% Frits Wenneker (http://www.howtotex.com)
%
% License:
% CC BY-NC-SA 3.0 (http://creativecommons.org/licenses/by-nc-sa/3.0/)
%
% Modificato per l'italiano da:
% Giacomo Roversi
% Arianna Coppola
%
%%%%%%%%%%%%%%%%%%%%%%%%%%%%%%%%%%%%%%%%%

%----------------------------------------------------------------------------------------
%	PACKAGES AND OTHER DOCUMENT CONFIGURATIONS
%----------------------------------------------------------------------------------------

\documentclass[DIV=12, BCOR=0pt]{scrartcl}  % A4 paper and 11pt font size {scrartcl}
% \usepackage{geometry}
% \geometry{a4paper,top=1.5cm,bottom=2.5cm,left=2cm,right=2cm} %,heightrounded,bindingoffset=0mm}

\usepackage[default]{cantarell}
% \usepackage{fourier}        % Use the Adobe Utopia font for the document
% \usepackage{lmodern}        % dimensioni compatibili con Cantarell
% \usepackage{beton}          % problemi con maiuscoletto e grassetto
% \usepackage{euler}
% \usepackage{concmath}
% \renewcommand*\familydefault{\sfdefault} 

\usepackage[T1]{fontenc}      % Use 8-bit encoding that has 256 glyphs
\usepackage{amsmath,amsfonts,amsthm, amssymb}    % Math packages
\usepackage{mathrsfs} 
\usepackage[italian]{babel}   % Adatta LaTeX alle convenzioni tipografiche italiane.
\usepackage[utf8x]{inputenc}  % Consente l'uso caratteri accentati italiani
\usepackage[dvips]{graphicx}  %per poter inserire le immagini

\usepackage{tabu, tabularx}
\usepackage{sectsty}          % Allows customizing section commands
\allsectionsfont{\normalfont\scshape} % Make all sections the default font and small caps

% \usepackage{fancyhdr}       % Custom headers and footers
% \pagestyle{fancyplain}      % Makes all pages in the document conform to the custom headers and footers
% \fancyhead{}                % No page header
% \fancyfoot[L]{}             % Empty left footer
% \fancyfoot[C]{}             % Empty center footer
% \fancyfoot[R]{\thepage}     % Page numbering for right footer
% \renewcommand{\headrulewidth}{0pt}    % Remove header underlines
% \renewcommand{\footrulewidth}{0pt}    % Remove footer underlines
% \setlength{\headheight}{13.6pt}       % Customize the height of the header
% 
\setlength\parindent{0pt}     % Controls indentation of the paragraphs (default=0)
\linespread{1.1}

% \usepackage{fixltx2e}       %per poter inserire apici e pedici
\usepackage[font=small,format=hang,labelfont={sf,bf}]{caption} %per personalizzare le didascalie delle immagini
% \usepackage{emptypage}      %eventuali pagine bianche senza né testa né piè di pagina
\usepackage{float}            %per posizionare le immagini dove voglio io
\usepackage{subfig}           %per inserire sottofigure
\usepackage{rotating}         %per ruotare le immagini
\usepackage{booktabs}         %per migliorare esteticamente le tabelle
% \usepackage{bookmark}
\usepackage[bookmarks]{hyperref}
\usepackage{color,listings}
\usepackage{multicol, makeidx}
\bibliographystyle{harward}


\usepackage{physics}
\usepackage{tikz}

% \usepackage{ragged2e}

% Arianna
% \usepackage{fullpage}
\usepackage{enumerate}
% \usepackage{wrapfig}
% \usepackage{sidecap}
\usepackage{siunitx}
\usepackage{eurosym}
\graphicspath{{immagini/}}

% CUSTOM COMMANDS
\newcommand{\horrule}[1]{\rule{\linewidth}{#1}}   % Creates horizontal rule command with 1 argument of height
\newcommand*{\vimage}[1]{\vcenter{\hbox{\includegraphics[width=3.45cm]{#1}}}}


%----------------------------------------------------------------------------------------
%	TITLE SECTION
%----------------------------------------------------------------------------------------

\title{
    \normalfont \small %\footnotesize
    \scshape{Università di Bologna - Laurea Magistrale in Fisica del Sistema Terra - A. A. 2018/2019} \\ [7pt]   % Your university, school and/or department name(s)
    \horrule{0.495pt} \\[0.3cm]  % Thin top horizontal rule
    \Large IDROLOGIA E RISCHIO IDRAULICO\\
    \huge Rapporto delle esercitazioni in R    % The assignment title
    \horrule{0.495pt} \\         % Thick bottom horizontal rule
}

\author{
	Giacomo Roversi\\ 						% Your name
	\normalsize\itshape{giacomo.roversi2@studio.unibo.it}
}
\date{\large\today}  						% Today's date


%----------------------------------------------------------------------------------------
%	MAIN SECTION
%----------------------------------------------------------------------------------------


\begin{document}
    \maketitle       % Prints the title
    
    \section{Introduzione}
    
    \subsection{Disdrometria e relazioni empiriche}
    
    \newpage
    \section{Analisi dei dataset}
    \subsection{Relazione tra velocità e diametro della goccia}

    \newpage
    \section{Conclusioni}

\end{document}
    
% VADEMECUM

% riferimento alla figura 
% In Fig. \ref{fig:scatterplot}

% formula in linea
% \[V \left( D \right) \;\;=\;\; 9.65 - 10.3 \cdot e^{-0.6 \cdot D }\]
% 
% 
%     \begin{figure}[!htb]
%         \centering
%         \fbox{\includegraphics[width=0.5\linewidth]{../Es01/immagini/tirante.png}}
%         \caption{Grandezze di interesse nella sezione fluviale.}
%         \label{fig:sezione}
%         \end{figure}
% 
%         
%     \begin{table}[!htb]
%         \footnotesize 
%         \centering
%         \begin{tabu} to \textwidth {X[0.5,c]X[1,c]X[1,c]X[1,c]X[1,c]X[1,c]X[1,c]X[1,c]X[1,c]}
%             \toprule
%             &\multicolumn{2}{c}{Campagna 1} & \multicolumn{2}{c}{Campagna 2} & \multicolumn{2}{c}{Campagna 3} & \multicolumn{2}{c}{Campagna 4} \\
%             \#&h & Q & h & Q & h & Q  & h  & Q  \\
%             &\lbrack\si{\m}\rbrack & \lbrack\si{\m^3\per\second}\rbrack & \lbrack\si{\m}\rbrack & \lbrack\si{\m^3\per\second}\rbrack & \lbrack\si{\m}\rbrack & \lbrack
%             \si{\m^3\per\second}\rbrack & \lbrack\si{\m}\rbrack & \lbrack\si{\m^3\per\second}\rbrack \\
%             \midrule
%             1&0.6104&25.800&0.5126&6.540&0.0365&0.083&0.1113&1.530 \\
%             2&0.9856&58.100&0.8603&22.200&0.3395&7.800&0.1483&1.530 \\
%             3&0.2066&2.890&0.0727&0.618&0.9601&31.100&0.9728&24.600 \\
%             4&0.4983&19.300&0.0418&0.161&0.5941&13.100&0.0874&0.557 \\
%             5&0.0088&0.0097&0.8269&35.800&0.8374&33.700&0.7173&36.900 \\
%             6&&&&&0.3807&6.710&0.3896&4.700 \\
%             7&&&&&0.3150&6.010&0.5195&17.200 \\
%             8&&&&&0.8747&39.800&0.9033&33.200 \\
%             9&&&&&0.0711&0.557&0.7220&31.100 \\
%             10&&&&&0.8885&31.700&0.6156&26.200 \\
%             11&&&&&0.9490&45.500&0.7807&43.800 \\
%             12&&&&&0.4826&14.500&0.5660&12.800 \\
%             13&&&&&0.2228&4.190&0.5000&13.600 \\
%             14&&&&&0.5578&19.100&0.1918&3.570 \\
%             15&&&&&0.6692&22.600&0.3077&5.590 \\
%             \bottomrule
%             \end{tabu}
%         \caption{Misure di portata e dei relativi tiranti ricavate da quattro distinte campagne di misura.}
%         \label{tab:misure.input}
%         \end{table}

% \begin{table}[!htb]
% 		\centering
% 		\begin{tabular}{lcc}
% 			\textbf{Variabile}&\vline&\textbf{Varianza}\\ 
% 			\hline
% 			u&\vline&0.01\\
% 			v&\vline&0.01\\
% 			w&\vline&0.01\\
% 			T&\vline&0.001\\
% 		\end{tabular}
% 		\caption{Varianza delle quattro variabili senza perturbazioni alla temperatura della stanza.}
% 		\label{tab:var_1}
% 	\end{table}	
    
% figure sovrapposte (2 per pagina)
%     \begin{figure}[!htb]
%         \centering
%         \includegraphics[width=0.89\linewidth]{../Scatterplot_DV/strat.png}
%         \includegraphics[width=0.89\linewidth]{../Scatterplot_DV/conv.png}
%         \caption{Scatterplots tra velocità di caduta e diametro della goccia per un evento stratiforme (in alto) e uno convettivo (in basso). Dataset da disdrometro a imaging ottico 2DVD.}
%         \label{fig:scatterplot}
%     \end{figure}
    
% figure affiancate e sovrapposte (4 per pagina)
% 	\begin{figure}[!htb]
% 		\centering
% 		\includegraphics[width=0.495\linewidth]{../Size_Distribution/strat1.png}
% 		\includegraphics[width=0.495\linewidth]{../Size_Distribution/strat2.png}
% 		\includegraphics[width=0.495\linewidth]{../Size_Distribution/strat3.png}
% 		\includegraphics[width=0.495\linewidth]{../Size_Distribution/strat4.png}
% 		\includegraphics[width=0.495\linewidth]{../Size_Distribution/strat5.png}
% 		\includegraphics[width=0.495\linewidth]{../Size_Distribution/strat6.png}	
% 		\caption{Distribuzioni dimensionali delle gocce durante una precipitazione stratiforme per istanti successivi a distanza di 4 minuti. In legenda i parametri delle dei fit.}
% 		\label{fig:stratdistr}
% 	\end{figure}

    
