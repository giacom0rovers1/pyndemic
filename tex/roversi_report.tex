% Giacomo Roversi - Remote Sensing
%\documentclass[DIV=14, paper=a4, fontsize=10pt]{scrartcl}
\documentclass[DIV=12, BCOR=0pt]{scrartcl}  % A4 paper and 11pt font size {scrartcl}
\usepackage[utf8x]{inputenc}
\usepackage[english]{babel}
%\usepackage[default]{raleway}
\usepackage[default]{cantarell}
\usepackage[cm]{sfmath}
\usepackage{amsmath}
\usepackage[usenames,dvipsnames]{xcolor}
\usepackage{hyperref}
\usepackage[round]{natbib}
\usepackage{tabu}
\usepackage{booktabs} % \toprule \midrule \bottomrule
\usepackage{physics}
\usepackage{siunitx}
\usepackage{eurosym}
\usepackage{graphicx}


\definecolor{linkcolour}{rgb}{0,0.2,0.6}
\hypersetup{colorlinks,breaklinks,urlcolor=linkcolour,linkcolor=linkcolour}  


\begin{document}
%	
%	\begin{flushright}
%	% Author	
%	{\large Giacomo Roversi} \\ 
%	{\small \href{mailto:giacomo.roversi2@studio.unibo.it}{giacomo.roversi2@studio.unibo.it} }\\
%	\vskip0.2in
%  % Date
%  {\large November 12, 2020}
%	\end{flushright}
%
%	\begin{flushleft}
%	\vskip0.2in
%	% Title
%	{\huge \textbf{Numerical simulations of influenza-like epidemic processes over complex networks}} 
%	\vskip0.2in
%	% Subtitle
%	{\Large \color{gray}
%	Reti complesse A.A. 2017-2018  \\
%	Università di Bologna - Laurea Magistrale in Fisica del Sistema Terra}
%	\end{flushleft}



\begin{flushright}
	% Author	
	{\large Giacomo Roversi} \\ 
	{\small \href{mailto:giacomo.roversi2@studio.unibo.it}{giacomo.roversi2@studio.unibo.it} }\\
	\vskip0.15in
	% Date
	{\large November 12, 2020}
\end{flushright}

\begin{flushleft}
	% 
	{\Large \color{gray}
		Reti complesse A.A. 2017-2018 - Prof. Daniel Remondini  \\
		Università di Bologna - Laurea Magistrale in Fisica del Sistema Terra}


	\vskip0.2in
	% Title
	{\huge\textbf{Numerical simulations of influenza-like epidemic processes over complex networks}} 
	\vskip0.33in
	
\end{flushleft}
%	\section*{Abstract}
	
	\section{Introduction}
	\label{sec:intro}
	This work is based on \citet{Pastor-Satorras} %  Pastor-Satorras et al. (2015) 
	and aims to analyse numerically various scenarios of contagion, inspired to the current COVID-19 pandemic, exploiting NetworkX and NDlib python libraries.	
 Firstly, theoretical fundamentals and deterministic predictions of the SEIR model are presented. These predictions are then compared with discrete-time numerical simulations over complex networks of increasing resemblance to the real world, in order to show different aspects of the contagion dynamics. Finally, some mitigation and immunization strategies are tested.
  
  \section{The SEIR epidemic model}
  The model ... 
  
  Its basic implementation assumes homogeneous mixing of the population and consequent completely random interactions. Under this assumptions the transitions between compartments are regulated by a set of deterministic differential equations. 
  
  \begin{figure}[h]
  	\centering
  	\includegraphics[width=0.75\linewidth]{../SEIR_2.png}
  	\caption{}
  	\label{}
  \end{figure}

	Parameters are computed (R0, K, describe...)
	
  
  \section{Complex networks}
  Translate: \textit{L'assunzione di omogeneita` e` sicuramente inadeguata per sistemi reali che presentano forte eterogeneita` di composizione e di rapporti relazionali. Si affronta la modellizzazione di questa eterogeneita` relazionale tramite reti complesse sulle quali riprodurre i modelli di contagio per step di interazione individuo-individuo discretizzati. In una rappresentazione discreta su network, i parametri del modello acquisiscono il valore di probabilita` di transizione (quindi su reti piccole si evidenzia l'effetto di fluttuazioni random, necessario approccio ensemble).  }
  
  
  Approach: recreate the same analysis work flow of the deterministic model
  Nodes = individuals, links = way of interaction (contagion)
  
  \subsection{Random network}
  Erdos Renyi
  Contagion on random network evolves similarly to the deterministic ODE problem. 
  
  Smoothing of the positives (Exposed + Infected) curve is needed to avoid contamination from numerical and most importantly network noise. 
  
  Network noise is strictly linked to the random nature of the network, locally there are deviations from the average connectivity which make the epidemic evolve ad different speeds
  
  
  \subsection{Lattice}
  Ring
  The contagion is strongly inhibited, starts exponentially slower than the rest, Rt(0) most similar to predicted R0,
  then stops (no, oscillates around R=1, still going) with plenty of susceptible nodes still available but not reachable. Try with more days.
  
  
  \subsection{Small-World}
  Watts Strogatz
  Starts similar to lattice, evolves similar to random. GOOD
  
  \subsection{Scale free}
  Barabasi-Albert
  Initial boost is the fastest, K=0.006 and Td is less than 12 days. Look at the degree structure..
  
  
  \subsection{Scale free with clustering}
  Holme-Kim
  The initial growth is almost exponential and K is high but after that the pandemic slows down extremely fast
  
  
  
  
  
  
  \section{Immunization and mitigation strategies}
  \subsection{Lockdown and quarantine}
  \subsection{Vaccines}
  \subsubsection{Random (failure)}
	\subsubsection{Targeted (attack)}

	\section{Conclusions}

		
%  \small
  \footnotesize
	\bibliographystyle{plainnat} %  % apalike
	% Apalike non mostra i DOI, plainnat o abbrvnat mettono i nomi prima dei cognomi
	\bibliography{networks}

\end{document}



%%% Vademecum %%%
Bibliography:
\bibliographystyle{plainnat} %  % apalike
% Apalike non mostra i DOI, plainnat o abbrvnat mettono i nomi prima dei cognomi
\bibliography{networks}

Citations:
\citep{bibid}
\citet{bibid}

Math expressions:
\begin{align}
	a = b
\end{align}

Tables:
\begin{table}[h]
	\centering
	\begin{tabu} to \textwidth {X[1,l]X[1,c]X[1,c]X[1,c]} 
		\toprule
		Product & Reference  & Latency  & Spatial \\
		&time step (min)  &  (min)  &   resolution (km) \\ 
		\midrule
		CML & 15 &20 & 5 \\
		Radar raw & 5 & 15 & 1 \\
		Radar adj. & 60 & 60 &  1 \\
		Raingauges raw & 60 & 60 &  - \\
		ERG5 & 60 & 1440 & 5 \\ 
		\bottomrule
	\end{tabu}
	\caption{Latency and spatial and temporal sampling of the considered precipitation products.}
	\label{tab:latency}
\end{table}

Figures:
\begin{figure}[!htb]
	\centering
	\includegraphics[width=0.75\linewidth]{../Es01/immagini/dati4.png}
	\caption{Portate in funzione dei tiranti per le 4 campagne di misura, con incertezze al 25\%.}
	\label{fig:dati}
\end{figure}
